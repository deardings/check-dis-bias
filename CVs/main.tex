\documentclass[11pt,a4paper,sans]{moderncv} % Font sizes: 10, 11, or 12; paper sizes: a4paper, letterpaper, a5paper, legalpaper, executivepaper or landscape; font families: sans or roman

% ModernCV themes
\moderncvstyle{classic} % Options: 'casual' (default), 'classic', 'oldstyle', 'banking'
\moderncvcolor{blue} % Options: 'blue' (default), 'orange', 'green', 'red', 'purple', 'grey', 'black'

% Character encoding
\usepackage{fontawesome5} % For icons like GitHub, LinkedIn, etc.
\usepackage{academicons} % For academic icons like ORCID, Google Scholar

% Adjust the page margins
\usepackage[scale=0.75]{geometry}
\setlength{\hintscolumnwidth}{3cm} % Adjust the width of the hints column

% Personal data
\name{Alex}{Chen}
\title{PhD Candidate in Computer Science}
\address{123 Quantum Leap Way}{Palo Alto, CA 94301}{USA}
\phone[mobile]{+1 (555) 123-4567}
\email{alex.chen.research@email.com}
\homepage{alexchenresearch.com} % Optional
\social[linkedin]{alexchen-phd} % Optional
\social[github]{alexchen-code} % Optional
\extrainfo{\aiOrcid{0000-0000-0000-0000} \quad \aiGoogleScholar{XYZ123ABC}} % Optional

% Optional: If you want to add a quote or a tagline
%\quote{"Striving to advance the frontiers of AI through innovative research."}

\begin{document}

\makecvtitle

\section{Education}
\cventry{2022--Present}{PhD in Computer Science}{Stanford University}{Stanford, CA, USA}{}{Expected Graduation: May 2026. Advisor: Prof. Evelyn Reed. Specialization: Machine Learning, Natural Language Processing.}
\cventry{2020--2022}{M.S. in Computer Science}{Stanford University}{Stanford, CA, USA}{}{\textit{Thesis: "Advancements in Few-Shot Learning for Specialized Domains"}}
\cventry{2016--2020}{B.S. in Computer Science}{University of California, Berkeley}{Berkeley, CA, USA}{}{\textit{Summa Cum Laude, Minor in Statistics}}

\section{Awards and Honors}
\cvitem{2023}{[COMPANY] PhD Fellowship (Fictive)}
\cvitem{2022}{Stanford Graduate Fellowship}
\cvitem{2020}{Outstanding Undergraduate Researcher Award, UC Berkeley}
\cvitem{2019}{Dean's High Honors List (Multiple Semesters), UC Berkeley}
\cvitem{2018}{ACM ICPC Regional Competitor}

\section{Academic Research Experiences}
\cventry{Summer 2024}{Student Researcher (Anticipated)}{[COMPANY] Research}{Mountain View, CA (or Remote)}{}{Working with [COMPANY] research scientists on open-ended exploratory projects in advanced machine learning algorithms and their applications.}
\cventry{2022--Present}{Graduate Research Assistant}{Stanford AI Lab (SAIL)}{Stanford, CA, USA}{}{
\begin{itemize}
    \item Conducting research on robust and interpretable Natural Language Understanding models, focusing on cross-lingual transfer learning and commonsense reasoning.
    \item Developing novel deep learning architectures for efficient information extraction from large-scale unstructured text data.
    \item Collaborating with a team of 5 researchers, contributing to algorithm design, experimental setup, and manuscript preparation.
    \item Mentored junior PhD students in experimental design and paper writing.
\end{itemize}}
\cventry{2021--2022}{Master's Thesis Researcher}{Stanford NLP Group}{Stanford, CA, USA}{}{
\begin{itemize}
    \item Investigated and developed few-shot learning techniques for text classification in specialized domains with limited labeled data.
    \item Implemented and benchmarked various meta-learning algorithms, achieving state-of-the-art results on several internal datasets.
\end{itemize}}
\cventry{2019--2020}{Undergraduate Research Assistant}{Berkeley AI Research (BAIR) Lab}{Berkeley, CA, USA}{}{
\begin{itemize}
    \item Assisted Prof. Ben Carter with projects on adversarial attacks and defenses for computer vision models.
    \item Contributed to the development of a new dataset for evaluating model robustness.
\end{itemize}}

\section{Publications}
\cvitem{}{\textbf{Chen, A.}, Lee, S., \& Reed, E. (2024). "Cross-Lingual Alignment Strategies for Zero-Shot Commonsense Reasoning." \textit{Proceedings of the Annual Meeting of the Association for Computational Linguistics (ACL)}. (To Appear)}
\cvitem{}{\textbf{Chen, A.}, \& Reed, E. (2023). "Efficient Transformers for Low-Resource Language Understanding." \textit{Findings of Empirical Methods in Natural Language Processing (EMNLP)}. pp. 1234-1245.}
\cvitem{}{Kim, J., \textbf{Chen, A.}, \& Davis, M. (2023). "Probing the Limits of Pre-trained Language Models in Scientific Literature." \textit{Workshop on Scholarly Document Processing, NeurIPS}.}
\cvitem{}{\textbf{Chen, A.} (2022). "Advancements in Few-Shot Learning for Specialized Domains." \textit{Stanford University M.S. Thesis Archive}.}
\cvitem{}{Carter, B., \textbf{Chen, A.}, \& Zhao, L. (2020). "A Benchmark for Evaluating Robustness of Object Detection Models to Synthetic Perturbations." \textit{arXiv preprint arXiv:2005.XXXXX}.}

\section{Industry Experience}
\cventry{Summer 2023}{Machine Learning Research Intern}{Leading AI Startup X}{Palo Alto, CA, USA}{}{
\begin{itemize}
    \item Designed and implemented a novel recommendation system using graph neural networks, resulting in a 15\% improvement in user engagement metrics.
    \item Collaborated with software engineering teams to integrate the ML model into the production environment.
    \item Presented research findings and model performance to senior leadership.
\end{itemize}}

\section{Internship Experience}
\cventry{Summer 2019}{Software Engineering Intern}{Tech Solutions Inc.}{San Francisco, CA, USA}{}{
\begin{itemize}
    \item Developed and tested new features for a large-scale data analytics platform using Python and Java.
    \item Participated in agile development sprints, code reviews, and system design discussions.
    \item Gained experience with cloud computing platforms (AWS).
\end{itemize}}

\section{Teaching Experience}
\cventry{Spring 2023}{Teaching Assistant}{CS224N: Natural Language Processing with Deep Learning}{Stanford University}{}{
\begin{itemize}
    \item Led weekly discussion sections for 30+ students, explaining complex concepts and guiding them through assignments.
    \item Graded homework, quizzes, and exams, providing constructive feedback.
    \item Held office hours and assisted students with project development.
\end{itemize}}
\cventry{Fall 2022}{Teaching Assistant}{CS109: Introduction to Probability for Computer Scientists}{Stanford University}{}{
\begin{itemize}
    \item Assisted in preparing course materials and grading assignments.
    \item Conducted review sessions before midterms and final exams.
\end{itemize}}

\section{Speaker Experience}
\cventry{2023}{Presenter, "Efficient Transformers for Low-Resource Language Understanding"}{EMNLP 2023 Conference, Singapore}{}{}{}
\cventry{2023}{Guest Lecturer, "Introduction to Meta-Learning"}{Stanford CS330: Deep Multi-Task and Meta Learning}{}{}{}
\cventry{2022}{Poster Presenter, "Few-Shot Learning for Medical Text Classification"}{Bay Area NLP Meetup}{}{}{}

\section{Academic Service}
\cventry{2023--Present}{Reviewer}{ACL, EMNLP, NeurIPS, ICML Conferences}{}{}{}
\cventry{2023}{Volunteer}{NeurIPS 2023 Conference, New Orleans}{}{}{}
\cventry{2022--2023}{PhD Admissions Committee Student Representative}{Stanford Computer Science Department}{}{}{}
\cventry{2021}{Organizer}{Stanford AI Lab Seminar Series}{}{}{}

\section{DEI Panels}
\cventry{2024}{Panelist}{"Navigating Academia as an Underrepresented PhD Student"}{Stanford School of Engineering DEI Summit}{}{}{}
\cventry{2023}{Participant}{"Building Inclusive Research Environments" Workshop}{Stanford University}{}{}{}

\section{News and Other Coverage}
\cventry{2023}{"Stanford Researchers Tackle Language Barriers in AI"}{Stanford Engineering News (Fictive article mentioning our EMNLP paper)}{}{}{}
\cventry{2020}{"Berkeley Undergrad Shines in AI Robustness Research"}{The Daily Californian (Fictive article about BAIR Lab project)}{}{}{}

\section{Other Activities}
\cventry{2022--Present}{Member}{Stanford Alpine Club, Stanford Cycling Team}{}{}{}
\cventry{2016--2020}{Volunteer Tutor}{Berkeley High School (Mathematics and Computer Science)}{}{}{}
\cvitem{}{Proficient in English (Fluent), Mandarin (Conversational)}
\cvitem{}{Programming Languages: Python (Expert), Java (Proficient), C++ (Intermediate), JavaScript (Basic)}
\cvitem{}{Frameworks/Libraries: TensorFlow, PyTorch, Scikit-learn, Pandas, NumPy, Hugging Face Transformers}
\cvitem{}{Developer Tools: Git, Docker, SLURM, AWS, GCP}

\closesection{} % Ensure the last section is properly closed

\end{document}